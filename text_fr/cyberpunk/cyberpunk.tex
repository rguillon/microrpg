%! Author = renaud.guillon@gmail.com
%! Date = 19/04/2020

\date{} % clear date
\maketitle{}
\tableofcontents

\section{Présentation}

Le monde de Cyberpunk 2020 est un univers sombre.
L'humanité côtoie le meilleur de la technologie et le pire de l'homme.
De transformations génétiques en cybernétiques, l'homme perd peu à peu son humanité.
Les grandes villes grouillent de bandes en tout genre, mais la société reste contrôlée par des puissants.
De grandes entreprises qui, à coups d'opérations plus au moins légales,
de guerres ou autres, ont la main mise sur le pouvoir.
Les employés de ces corpos travaillent à vie et restent généralement fidèles et dévoués à leurs entreprises.
Les gouvernements, visiblement torpillés par les corporations, à coups de scandales, de pots de vin,
sont devenus des marionnettes entre leurs mains.
Les intrigues intra-corporatistes impliquent l'emploi de forces vives.
Et c'est généralement là que les choses commencent. Victime ou employé,
innocent ou rebelle chacun apporte un peu de ses talents. Guerriers pour les nouveaux soldats
(les solos et les nomades), techniques pour les ingénieurs et bidouilleurs de cybernétique en tout genre
(les techies et les medtechies), relationnels et mise en contact de personnes ou de biens
(rockerboy, media, fixer, corpo), d'interpolation du réseau avec les pirates informatiques (les netrunners).
Il reste encore quelques défenseurs d'un ordre oublié qui sont les policiers (les cops) des années 2020.
Dans ce marasme cynique, chacun essaye de survivre tout en étant sur la limite,
l'ultime frontière qui vous fera basculer du mauvais côté.

